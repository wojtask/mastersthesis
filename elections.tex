\chapter{Elections and Election-Related Problems} \label{ch:elections}

In order to study elections from the point of view of computational complexity theory, we need to transform the intuitive sense of what elections and voting systems are into a precise mathematical model.
The current chapter includes formal definitions of these notions, and specifications of control problems we will study under several voting systems.
We also make an overview of the most common voting rules along with the demonstration of how they work for some specified collections of candidates and voters.
We point the reader to \cite{arrow} for general reference.

\section{Elections and Voting Systems} \label{sec:elections}

We begin by formalizing the central notion of this paper---an election.

\begin{definition}
    An election $E$ is an ordered pair $(C,V)$ where $C$ is a finite non-empty set of candidates and $V$ is a finite non-empty multiset of voters, where each voter is represented by a total order over $C$.
\end{definition}

In this work we use $m$ to denote the number of candidates in $C$ and $n$ to denote the number of voters in $V$, and assume that $C=\{c_0,c_1,\dots,c_{m-1}\}$.
Voters vote on the candidates by setting their preference lists, i.e., by putting the candidates in order from the most preferred one to the most despised one.
Thus, $V=\{\succ_0,\succ_1,\dots,\succ_{n-1}\}$ is a multiset over the set of all total orders over $C$.
In most cases we study, such definition of election is sufficient, although in many situations we do not have the full information about voters' preferences.
We may have only partial knowledge and do not know what the rest of the voter's preference is.
Thus, every element of $V$ can be treated as a partial order over $C$ and in this case we say $E=(C,V)$ is a \emph{partial election}.

\begin{Example} \label{ex:election1}
	Let us consider an election $E=(C,V)$, where:
	\begin{align*}
		C &= \{c_0,c_1,c_2\}, \\
		V &= \bigl\{c_1\succ_0c_2\succ_0c_0,\;c_2\succ_1c_1\succ_1c_0,\;c_0\succ_2c_2\succ_2c_1,\;c_1\succ_3c_2\succ_3c_0\bigr\}.
	\end{align*}
	We have $m=3$ candidates and $n=4$ voters, where the first and the fourth voter provide the same ranking:
	They prefer candidate $c_1$ over $c_2$, and $c_2$ over $c_0$.
\end{Example}

Once we have an election, we need some formal rule---or protocol---for determining the winner(s) of the election.
We call this rule a \emph{voting system}.
Consider the following definition.

\begin{definition}
A voting system $\mathcal{E}$ is defined to be a function mapping an election $E=(C,V)$ to a set $W\subseteq C$ of candidates who win the election $E$.
\end{definition}

We also call $\mathcal{E}$ a voting rule, a voting protocol, or a voting scheme.
The members of the set $W$ from the definition above are called $\mathcal{E}$-winners---or simply winners---of election $E$ when a voting system is known from a context.
The set $W$ may be empty if there are no winners of the election, may be a singleton---we will then say that the election has a \emph{unique winner}---or may have more than one element.

There are very many voting systems.
We, in particular, study some of the so-called \emph{scoring protocols} and some of \emph{Condorcet-consistent rules}.

Scoring protocols are an important class of voting systems.
Given $m$ candidates, each scoring protocol is defined by a \emph{scoring vector} $\boldsymbol\alpha=(\alpha_1,\alpha_2,\dots,\alpha_m)$ satisfying $\alpha_1\ge\alpha_2\ge\cdots\ge\alpha_m\ge0$.
When a candidate is placed on the $i$th position in a voter's preference order, he receives $\alpha_i$ points.
After summing up the scores granted by each voter, the candidate(s) with the highest score are the winners of the election.
Given a scoring system, we write $\score{E}(c_j)$ to mean the score of candidate $c_j$ in election $E$.
The most common scoring systems are:
\begin{Enumerate}
	\item \emph{plurality voting}, with the scoring vector $\boldsymbol\alpha_{\id{Plurality}}=(1,\overbrace{0,\dots,0}^{m-1})$,
	\item \emph{veto voting}, with the scoring vector $\boldsymbol\alpha_{\id{Veto}}=(\overbrace{1,\dots,1}^{m-1},0)$, and
	\item \emph{Borda count}, with the scoring vector $\boldsymbol\alpha_{\id{Borda}}=(m-1,m-2,\dots,0)$.
\end{Enumerate}

Let us consider once again election $E$ from Example~\ref{ex:election1}, and determine which candidates win under the three voting systems defined above.
In plurality rule $c_1$ receives 2 points, while other candidates get 1 point each, so clearly $c_1$ is the unique winner of $E$.
Formally, we write $\id{Plurality}(E)=\{c_1\}$.
Under veto rule, $c_2$ receives 4 points, $c_1$ receives 3 points, and $c_0$ receives only 1 point, so $\id{Veto}(E)=\{c_2\}$.
In Borda voting we have that both $c_1$ and $c_2$ get 5 points, while $c_0$ gets only 2 points, thus $\id{Borda}(E)=\{c_1,c_2\}$.
This example shows that our choice of the voting rule is very important, because under different rules we may get different sets of winners.
The other thing to notice is that the more voters rank some candidate low, the less chance this candidate has to be a winner.
This is a direct consequence of the fact that the scoring vector is a non-increasing sequence.

Another type of voting system is \emph{approval voting}.
Its structure is slightly different from the other systems, as we do not treat voters' preference orders as total or partial orders but as subsets of $C$.
Each voter $v_i\in V$ provides a subset $A_i\subseteq C$ of candidates that he approves of.
A candidate $c\in C$ is a winner, when he appears in the largest number of such subsets.

\begin{Example} \label{ex:election2}
    Let $E=(C,V)$, where:
	\begin{align*}
		C &= \{c_0,c_1,c_2\}, \\
		V &= \bigl\{\underbrace{\{c_0,c_1,c_2\}}_{A_0},\underbrace{\{c_0,c_1,c_2\}}_{A_1},\underbrace{\{c_0\}}_{A_2},\underbrace{\{c_0,c_1,c_2\}}_{A_3}\bigr\}.
	\end{align*}
	Voters' preferences are similar to those in Example~\ref{ex:election1}, but they do not provide the order of candidates in each vote.
	Each voter specifies the set of candidates that he approves of.
	We can see that every voter but the third approves of all the three candidates, and the remaining one approves only of $c_0$.
	With this configuration, $c_0$ gets 4 points, and other candidates get 3 points each.
	Thus $\id{Approval}(E)=\{c_0\}$.
	This example shows that it is possible to secure victory of a candidate even though it might seem unlikely under other voting schemes, since $c_0$ is not a winner of the similar election from Example~\ref{ex:election1} under every other voting scheme considered in this section.
\end{Example}

There is also a modified version of this election system, called \emph{$k$-approval voting} (for integer $k$ such that $1\le k\le m$) in which we demand that every voter chooses exactly $k$ candidates, i.e., $|A_i|=k$, for each integer $i$, $1\le i\le n$.
$k$-approval can also be viewed as a scoring protocol with scoring vector $\boldsymbol\alpha_{k\text{-}\!\id{Approval}}=(\underbrace{1,\dots,1}_k,\underbrace{0,\dots,0}_{m-k})$.
Note that the case $k=m$ is trivial, as the whole $C$ is the set of winners, if $k=1$, this system is equivalent to plurality voting and if $k=m-1$ then we have veto.
When $1\le k\le m-1$, $(m-k)$-approval is often called \emph{$k$-veto voting}.

The second class of voting rules that we study here are those that consider results of pairwise comparisons of candidates.
An important and well studied rule is \emph{Condorcet voting} in which $c\in C$ is a winner if and only if for each $c'\in C-\{c\}$, $c\succ_i c'$ for more than $n/2$ of preference orders $\succ_i$ from $V$.
There can be at most one winner under Condorcet's rule and he is called the \emph{Condorcet winner}.
We define $\beaten{E}(c_j,c_k)$ to be the number of preference orders $\succ_i$ from $V$, such that $c_j\succ_i c_k$.
Thus, $c$ is a Condorcet winner exactly, if $\beaten{E}(c,c')>\beaten{E}(c',c)$ for each $c'\in C-\{c\}$.

In Example~\ref{ex:election1} there is no Condorcet winner as $c_1$ wins with $c_2$ in 2 votes, $c_2$ wins over $c_1$ also in 2 votes, and $c_0$ wins over both $c_1$ and $c_2$ only in 1 vote.
Therefore, we can write $\id{Condorcet}(E)=\emptyset$.

The last voting system we consider in this work, called \emph{maximin voting}, is based on scoring while making use of some of the ideas from Condorcet rule.
In maximin system, the score of candidate $c\in C$ is defined as $\min_{c'\in C-\{c\}}\beaten{E}(c,c')$.
The score of candidate $c_0$ from Example~\ref{ex:election1} is 1, and the score of $c_1$ and $c_2$ is 2, thus $\id{Maximin}(E)=\{c_1,c_2\}$.

% The winner in \emph{Dodgson voting} is the candidate that becomes a Condorcet winner by fewest swaps of adjacent candidates in voters' preferences.
% In \cite{bartholdi2} Bartholdi, Tovey, and Trick prove that the winner determination and the question of which of the two candidates have greater score to be \NPclass-hard problems.
% Further studies of Hemaspaandra, Hemaspaandra, and Rothe \cite{hemaspaandra2} show that both of these problems are $\Theta^p_2$-complete.
% In \cite{bartholdi2} the authors also consider the problem in which one asks if a distinguished candidate has score at most some nonnegative constant, and they prove it is \NPclass-comptete.

% Because of such issues, \emph{simplified Dodgson voting}, where one can say who is a winner in polynomial time, is often considered.
% Let $\deficit{E}(c_j,c_k)$ be the minimum number of votes in election $E$ in which $c_j$ should be ranked ahead of $c_k$, so that $c_j$ wins the head-to-head contest with $c_k$.
% The winner of simplified Dodgson voting is a candidate $c\in C$ with the lowest score defined as $\score{E}(c)=\sum_{c'\in C-\{c\}}\deficit{E}(c,c')$.
% 
% For election $E$ from Example \ref{ex:election1}, we get the following values:
% \begin{align*}
%     \deficit{E}(c_0,c_1)=2, &\; \deficit{E}(c_0,c_2)=2, \qquad \score{E}(c_0)=4, \\
% 	\deficit{E}(c_1,c_0)=0, &\; \deficit{E}(c_1,c_2)=1, \qquad \score{E}(c_1)=1, \\
%     \deficit{E}(c_2,c_0)=0, &\; \deficit{E}(c_2,c_1)=1, \qquad \score{E}(c_2)=1.
% \end{align*}
% Thus, we have $\id{Simplified-Dodgson}(E)=\{c_1,c_2\}$.

\section{Control Problems in Elections and Their Counting Variants} \label{sec:control}

Control is an attack widely used in real-world elections.
Although many types of elections are performed secretly, i.e., nobody but the voter knows his own preferences, there are situations where some actor (or a group of actors) has partial or complete knowledge of all the votes.
In such situations this actor may seek to determine the way how the parameters of the election can be changed in order to achieve the desired outcome.
Political parties often keep track of the statistics and pre-election polls and develop advertisements directed at particular groups of people in order to convince them to vote.
Another example of control attack can be observed in song of the month contests.
A radio station may promote new songs by adding them to the list and increasing their chances for gaining popularity, hence increasing the producer income.

Given an election, we ask if it is possible to modify it in some way in order to make the distinguished candidate a winner (constructive control problem), or to prevent this candidate from being a winner (destructive control problem).
The types of such modifications will be precisely described later in this section, but in short, we can add or delete some candidates or voters.
Partitioning candidates or voters into a 2-round structure is also studied in the literature along with two different tie-breaking rules, but in this work we do not discuss these types of control.
The family of control problems has been formulated and studied by Bartholdi, Tovey, and Trick in \cite{bartholdi} and by Hemaspaandra, Hemaspaandra, and Rothe in \cite{hemaspaandra}.
These papers show that some of the modifications are computationally easy to perform in order to solve a problem, while the others are computationally hard, or even never lead to a solution.

In this work we focus on the counting variants of control problems rather than on the decision ones.
Therefore, we are interested not only in determining if a solution to the problem exists, but also in the existence of fast algorithms that count all the solutions to the given problem.

We now formally define all problems we study in this paper.
Every problem we consider has received a name which is made of the name of the election system $\mathcal{E}$; a short prefix defining the type, such as `AV' for adding voters; one of `${}_\textnormal{C}$' or `${}_\textnormal{D}$' subscripts denoting that the problem is either the constructive case or the destructive case; and a description part which is `\textsc{Control}' in each of them.
To distinguish them from the decision variants, we put a `\#' sign after election system part.
If we study the problem given some election system, say approval voting, we write `\!{\it Approval\/}' in the description part instead of $\mathcal{E}$.

\paragraph{Counting Variant of Control by Adding Candidates}
\begin{description}
	\item[Name:] \Control{AC}{C}
	\item[Given:] An election $E=(C,V)$, a set $C'$ of additional candidates with $C\cap C'=\emptyset$, a distinguished candidate $c\in C$, and a positive integer $k\le|C'|$.
	Each voter in $V$ provides preferences over $C\cup C'$.
	\item[Question:] How many sets $B\subseteq C'$ with $|B|\le k$ are there, such that $c$ is the unique $\mathcal{E}$-winner of election $E'=(C\cup B,V)$?
\end{description}

\paragraph{Counting Variant of Control by Deleting Candidates}
\begin{description}
	\item[Name:] \Control{DC}{C}
	\item[Given:] An election $E=(C,V)$, a distinguished candidate $c\in C$, and a positive integer $k\le m$.
	\item[Question:] How many sets $B\subseteq C$ with $|B|\le k$ are there, such that $c$ is the unique $\mathcal{E}$-winner of election $E'=(C-B,V)$?
\end{description}

\paragraph{Counting Variant of Control by Adding Voters}
\begin{description}
	\item[Name:] \Control{AV}{C}
	\item[Given:] An election $E=(C,V)$, a multiset $V'$ of additional voters, $V\cap V'=\emptyset$, with preferences over $C$, a distinguished candidate $c\in C$, and a positive integer $k\le|V'|$.
	\item[Question:] How many multisets $U\subseteq V'$ with $|U|\le k$ are there, such that $c$ is the unique $\mathcal{E}$-winner of election $E'=(C,V\uplus U)$?
\end{description}

\paragraph{Counting Variant of Control by Deleting Voters}
\begin{description}
	\item[Name:] \Control{DV}{C}
	\item[Given:] An election $E=(C,V)$, a distinguished candidate $c\in C$, and a positive integer $k\le n$.
	\item[Question:] How many multisets $U\subseteq V$ with $|U|\le k$ are there, such that $c$ is the unique $\mathcal{E}$-winner of election $E'=(C,V-U)$?
\end{description}

\bigskip
The instances of each problem above are tuples constructed from the elements listed in the ``Given'' parts.
For example, in \Control{AV}{C} each instance is of the form $(E,V',c,k)$.
The set of solutions in every case is of course the set of naturals, as in all counting problems.

The destructive versions of control problems differ only in the ``Question'' part, which demands that $c$ is \emph{not} the unique $\mathcal{E}$-winner of the appropriately defined election $E'$.
Moreover, in \Control{DC}{D} we prevent deleting $c$ (also assuming that $k<m$), since otherwise any voting system could have been trivially controlled.

When regarding decision variants of control problems we say a voting system is \emph{immune} to control if it is not possible for the chairman to change a distinguished candidate from a non-winner to a unique winner (or inversely in the destructive variant) by ma\-ni\-pu\-la\-ting the set of candidates or the multiset of voters.
Otherwise, we regard two distinct cases---if for an election system it is hard (i.e., \NPclass-complete) to find a way to control the election, the system is referred to as \emph{resistant}, and if this is always computationally easy (i.e., the problem is in \Pclass), the system is referred to as \emph{vulnerable}.
Of course, in the case of immunity the counting variant of control problem is trivially in \FPclass, as we always return 0; the other cases require a proof of either being in \FPclass\ or being \sharpPclass-complete.
As we have seen in the previous chapter, if some problem is \NPclass-complete it does not automatically mean that its counting variant is \sharpPclass-complete (however, it is typically the case).

\section{Winner Prediction Model} \label{sec:prediction}

We will now describe how winner prediction relates to control problems.
Let us say we are given a voting rule $\mathcal{E}$, a set $C$ of $m$ candidates, and a multiset $V$ of $n$ voters (for each voter we have perfect knowledge as to how he would vote).
We consider the situation where each candidate from a set $D$, $D\subseteq C$, may or may not participate in the election, and similarly, where each voter from a multiset $W$, $W\subseteq V$, may or may not actually vote.
For such uncertain election $E=(C,V)$ we ask for the probability that a designated candidate $c\in C$ is the unique winner under voting system $\mathcal{E}$.

Let us study an example scenario, showing the connection between problem \Control{AV}{C} and the appropriate type of winner prediction; the reader can imagine analogous settings for the remaining types of control.
Let us assume that set $C$ of candidates participating in the election is fixed (for example, because the election rules force all candidates to register well in advance).
We know that some multiset $V$ of voters will certainly vote (for example, because they have already voted and this information is public).
The collection of voters who have not decided to vote yet is $W$\!.
From some source (e.g., from prior experience) we have some probability distribution $P$ on the number of voters from $W$ that will participate in the election (from our perspective, each equal-sized subcollection of voters from $W$ is equally likely; different-sized collections may, of course, have different probabilities of participating in the election).

In other words, for each $i$, $0\le i\le|W|$, let $P(i)$ be the probability that $i$ voters from $W$ join the election (and assume that we have an easy way of computing this value) and let $Q(i)$ be the probability that a designated candidate $c$ wins under the condition that exactly $i$ voters from $W$ participate (assuming that each $i$-element subcollection of $W$ is equally likely).
Then, the probability that $c$ wins is simply given by:
\[
	P(0)Q(0)+P(1)Q(1)+\dots+P(|W|)Q(|W|).
\]
To compute $Q(i)$, we have to compute for how many collections $W$ of size exactly $i$, candidate $c$ wins, and divide it by $\binom{|W|}{i}$.
To compute for how many sets of size exactly $i$ candidate $c$ wins, we solve the corresponding \Control{AV}{C} problem for adding at most $i$ voters from $W$, then for adding at most $i-1$ voters from $W$, and then we subtract the results.

In our setting we are given two probability distributions, $P_C$ and $P_V$, defined on the set of all subsets of $C$ and, respectively, on the set of every subcollection of $V$.
We consider two possible scenarios:
\begin{Enumerate}
	\item The set of candidates is fixed (i.e., $P_C(C)=1$), but for each multiset of voters $V'$, $V'\subseteq V$, we have probability $P_V(V')$ that exactly the voters from $V'$ show up for the vote.
	\item The multiset of voters is fixed (i.e., $P_V(V)=1$), but for each set of candidates $C'$, $C'\subseteq C$, we have probability $P_C(C')$ that exactly the candidates from $C'$ participate in the election.
\end{Enumerate}

Naturally, out task would very quickly become computationally prohibitive if we did not assume anything about $P_C$ and $P_V$.
Of course, first of all we need to assume that both probabilities, $P_C$ and $P_V$, are polynomial-time computable.
We also would like to assume that functions $P_C$ and $P_V$ depend only on the cardinality of their arguments.
More formally, if $C_1$, $C_2\subseteq C$ and $|C_1|=|C_2|$, then $P_C(C_1)=P_C(C_2)$, and analogously for $P_V$.
In other words, we consider probability distribution regarding the number of candidates (the number of voters) participating in the election, but each same-cardinality subset is equally likely.

However, we will focus on some special situations that will allow us to easily transform our setting to counting variants of control problems.
In order to do this, we need to slightly weaken the assumption regarding the dependency only on the sizes of candidate/voter sets.
Often, we may have additional knowledge regarding the nature of possible changes in the candidate/voter set.
For example, the rules may be such that after a given point of time candidates can withdraw from the election but no new candidates can register.
Similarly, we may know that some votes have already been cast and cannot be withdrawn.
Thus, we refine our model to be the following:
We start with some candidates and voters already in the election, and we ask for the probability that a given candidate wins assuming that some random number of voters/candidates is added/deleted.

More formally, we start with an election $E=(C,V)$ and we consider the following situations:
\begin{Enumerate}
    \item The set of candidates $C$ is fixed, and each voter from $V$ will vote, but there is a multiset of unregistered voters $V'$, $V'\cap V=\emptyset$, that also may show up for the vote.
	\item The set of candidates $C$ is fixed, but some of the voters from $V$ may not show up for the vote.
	\item The multiset of voters $V$ is fixed, and each candidate from $C$ will participate in the election, but there is a set of unregistered candidates $C'$, $C'\cap C=\emptyset$, that also may participate.
	\item The multiset of voters $V$ is fixed, but some of the candidates from $C$ may not participate in the election.
\end{Enumerate}
In effect, after applying one of these scenarios to $E$ we obtain an election $E'$.
For example, in the second case above we consider $E'=(C,V-U)$, where $U\subseteq V$.
We also use a fixed upper bound for the number of candidates/voters that can be added/deleted in election $E$.

One can easily verify that the prediction of the winner in the presented setting reduces to the counting variants of control problems which have been defined in the previous section.

In the following chapters we study counting variant of control problems under several voting systems.
If for a voting system $\mathcal{E}$ a particular control problem is computationally easy, so is the corresponding type of winner prediction problem under $\mathcal{E}$.
