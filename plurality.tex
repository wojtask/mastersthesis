\chapter{Control Problems in Plurality Voting} \label{ch:plurality}

Under plurality voting, counting variants of both control by adding candidates and control by deleting candidates are \sharpPclass-complete, as we show it in the following theorems.
However, polynomial-time algorithms exist for the cases where we modify the collection of voters.

\section{Adding and Deleting Candidates} \label{sec:plcan}

The first problem we consider under plurality voting is constructive control by adding candidates.
Below we show that this problem is \sharpPclass-parsimonious-complete (the proof approach is inspired by a related result in \cite{faliszewski7}).

\begin{theorem} \label{th:placc}
    \PlControl{AC}{C} is \sharpPclass-parsimonious-complete.
\end{theorem}

\begin{proof}
In the first step we need to show that $\PlControl{AC}{C}\in\sharpPclass$.
From the definition of class \sharpPclass, we need give polynomial-time NDTM that accepts on as many computational paths as many solutions there are for our problem.
We can imagine a non-deterministic algorithm that, given an instance $I=(E,C',c,k)$, it simply checks all possible subsets of $C'$ of sizes at most $k$, if any of them consists of exactly those new candidates whose addition to $C$ makes $c$ a unique winner.
This requires only po\-ly\-no\-mial\-ly many steps because it can non-deterministically choose the next candidate from $C'$ in each step.

To prove that \PlControl{AC}{C} is \sharpPclass-parsimonious-hard, we show that there is a parsimonious reduction from the \sharpXthreeC\ problem (see Definition \ref{def:sharpXthreeC}).

Suppose we are given an instance of \sharpXthreeC, $(B,\mathcal{S})$, where $B=\{b_1,\dots,b_{3k}\}$ and $\mathcal{S}=\{S_1,\dots,S_r\}$ with $r\ge4$ and $k\ge2$.
For each $b_i\in B$, we set $\ell_i$ to be the number of sets in $\mathcal{S}$ that contain $b_i$.

We construct an election $E=(C,V)$, where $C=B\cup\{c,w\}$ is the set of registered candidates, and $V$ is a multiset of voters.
We also have a set $A=\{a_1,\dots,a_r\}$ of spoiler (unregistered) candidates.
Each candidate $a_i$ in $A$ corresponds to a set $S_i$ in $\mathcal{S}$.
Collection $V$ contains groups of voters with preferences over $C\cup A$ described below.
For each vote we specify up to two top-ranked candidates; the remaining candidates are ranked in arbitrary order in the `$\cdots$' part.
\begin{Enumerate}
	\item For each set $S_j\in\mathcal{S}$, for each $b_i\in S_j$, we have $2k$ votes $a_j\succ b_i\succ\cdots$.
	\item For each set $S_j\in\mathcal{S}$, we have 1 vote $a_j\succ c\succ\cdots$.
	\item We have $2rk+k-r+1$ votes $c\succ\cdots$.
	\item We have $2rk$ votes $w\succ\cdots$.
	\item For each $b_i\in B$, we have $2rk+2k-2k\ell_i$ votes $b_i\succ\cdots$.
\end{Enumerate}
Thus in election $E$ the scores of candidates are as follows:
\begin{Enumerate}
	\item $\score{E}(c)=2rk+k+1$,
	\item $\score{E}(w)=2rk$, and
	\item for each $b_i\in B$, $\score{E}(b_i)=2rk+2k$.
\end{Enumerate}
That is, the winners of plurality election $E$ are exactly the candidates in $B$.
We claim that every set $A'$, $A'\subseteq A$, such that $|A'|\le k$ and that $c$ is a unique winner of plurality election $E'=(C\cup A',V)$ corresponds one-to-one to a family $\mathcal{S'}$, $\mathcal{S'}\subseteq\mathcal{S}$, such that $|\mathcal{S'}|=k$ and $\bigcup\mathcal{S'}=B$ (i.e., $\mathcal{S'}$ is an exact set cover of $B$).

Let us fix set $A'$.
It corresponds to a family $\mathcal{S'}$ of 3-sets from $\mathcal{S}$ such that for each integer $j$, $1\le j\le r$, $\mathcal{S'}$ contains set $S_j$ if and only if $A'$ contains $a_j$.
It is easy to see that in election $E'=(C\cup A',V)$ the scores of candidates are as follows:
\begin{Enumerate}
	\item $\score{E'}(c)=2rk+k+1-|A'|$,
	\item $\score{E'}(w)=2rk$,
	\item for each $b_i\in B$, $\score{E'}(b_i)=2rk+2k-2k\,|\{a_j\in A':b_i\in S_j\}|$, and
	\item for each $a_j\in A'$, $\score{E'}(a_j)=6k+1$.
\end{Enumerate}
Assume that $c$ is a unique winner of election $E'$.
Then $c$'s score is the highest among the scores of other candidates.
Inequality $\score{E'}(c)>\score{E'}(a_j)$ holds for each $a_j\in A'$ under assumption that $r\ge4$.
In order to have $\score{E'}(c)>\score{E'}(w)$, we need to assume that $|A'|\le k$, and thus $\score{E'}(c)\ge2rk+1$.
For each $b_i\in B$ there is $\score{E'}(c)>\score{E'}(b_i)$, which holds if $\score{E'}(b_i)\le2rk$, so we have that $|\{a_j\in A':b_i\in S_j\}|\ge1$.
It means that each $b_i\in B$ belongs to at least one set $S_j$.
The score of each of the $3k$ candidates in $B$ has to be decreased and each $a_j\in A'$ corresponds to decreasing the score of exactly 3 candidates in $B$, because every $S_j$ contains exactly 3 distinct elements of $B$.
Thus, $A'$ corresponds to an exact set cover of $B$.

On the other hand, let us assume $A'$ corresponds to an exact set cover of $B$.
In such a case $|A'|=k$ because in an exact set cover of $3k$-element set we must have exactly $k$ 3-element sets.
Thus, $\score{E'}(c)=2rk+1$.
Since each $a_j\in A'$ corresponds to a set $S_j\in\mathcal{S}$ that contains 3 distinct members of $B$, for each $b_i\in B$ we have that $|\{a_j\in A':b_i\in S_j\}|=1$ and therefore $\score{E'}(b_i)=2rk<\score{E'}(c)$.
For each $a_j\in A'$ it holds $\score{E'}(a_j)=6k+1<2rk+1=\score{E'}(c)$ (since we assumed that $r\ge4$).
As we can see, $c$ has the greatest score, so it is a unique winner of election $E'$.
\end{proof}

In our further proofs in which we show reductions from \sharpXthreeC\ problem, we will not be as precise and we will be describing things less explicitly.
In particular, we assume that the reader can easily verify that a problem belongs to class \sharpPclass\ based on the method in the proof above.

Thanks to the proved fact, we can easily show \sharpPclass-completeness (in the sense of Krentel) of the destructive control by adding candidates.
In our proof the answer to the problem depends on its instance as well as on the number of solutions in the constructive case.
Although we do not show the problem's possible \sharpPclass-parsimonious-completeness, we do not claim it does not hold.

\begin{theorem} \label{th:placd}
	\PlControl{AC}{D} is \sharpPclass-metric-complete.
\end{theorem}

\begin{proof}
Let $I=(E,C',c,k)$ be an instance of \PlControl{AC}{C} and let $s_I$ be the number of solutions to $I$, i.e., the number of sets of at most $k$ candidates from $C'$ whose inclusion in election $E$ makes $c$ the unique winner.
Note, there are exactly $\sum_{i=0}^k\binom{|C'|}{i}$ sets $B$ such that $B\subseteq C'$ and $|B|\le k$, so we have that the answer to \PlControl{AC}{D} for instance $I$ is exactly $\sum_{i=0}^k\binom{|C'|}{i}-s_I$.

After theorem~\ref{th:placc}, every \sharpPclass-problem parsimoniously reduces to \PlControl{AC}{C}.
We also have that \PlControl{AC}{C} metrically reduces to \PlControl{AC}{D}, since the number of solutions to $I$ depends on $s_I$ and on $I$ itself.
Thus, every \sharpPclass-problem metrically reduces to \PlControl{AC}{D}.
\end{proof}

Let us now move on to control by deleting candidates in plurality elections.
The proof of the following theorem is also based on the similar one from \cite{faliszewski7}.

\begin{theorem} \label{th:pldcc}
    \PlControl{DC}{C} is \sharpPclass-parsimonious-complete.
\end{theorem}

\begin{proof}
It is clear that this problem belongs to \sharpPclass.
We prove that it is \sharpPclass-parsimonious-hard by showing a parsimonious reduction from \sharpXthreeC.

Let $(B,\mathcal{S})$ be an instance of the \sharpXthreeC\ problem, where $B=\{b_1,\dots,b_{3k}\}$ and $\mathcal{S}=\{S_1,\dots,S_r\}$ with $k\ge7$ and $r\ge1$.

We construct a \PlControl{DC}{C} instance as follows.
Let $A=\{a_1,\dots,a_r\}$ and let $E=(C,V)$ be an election, where $C=B\cup A\cup\{c\}$, and collection $V$ contains groups of voters with the following preferences.
For each vote we specify up to two top candidates and up to one ranked-lowest candidate; the remaining candidates are ranked in arbitrary order in the `$\cdots$' part.
\begin{Enumerate}
	\item For each set $S_j\in\mathcal{S}$, for each $b_i\in S_j$, we have 1 vote $a_j\succ b_i\succ\cdots\succ c$.
	\item For each set $S_j\in\mathcal{S}$, we have 1 vote $a_j\succ c\succ\cdots$.
	\item For each $b_i\in B$, we have $k-2$ votes $b_i\succ\cdots\succ c$.
\end{Enumerate}
In election $E$ we have the following scores:
\begin{Enumerate}
	\item $\score{E}(c)=0$,
	\item for each $b_i\in B$, $\score{E}(b_i)=k-2$, and
	\item for each $a_j\in A$, $\score{E}(a_j)=4$.
\end{Enumerate}
The winners of election $E$ are exactly the candidates in $B$.
We claim that every set $A'$, $A'\subseteq A$, such that $|A'|\le k$ and that $c$ is a unique winner of plurality election $E'=(C-A',V)$ corresponds one-to-one to a family $\mathcal{S'}$, $\mathcal{S'}\subseteq\mathcal{S}$, such that $|\mathcal{S'}|=k$ and $\mathcal{S'}$ is an exact set cover of $B$.

First, assume that $A'$ is a subset of $A$ such that $\{S_i\in\mathcal{S}:a_i\in A'\}$ is an exact set cover of $B$.
It is easy to see that $c$ is a unique winner of plurality election $E'=(C-A',V)$, because $\score{E'}(c)=k$, for each $b_i\in B$, $\score{E'}(b_i)=k-1$, and for each $a_j\in A-A'$, $\score{E'}(a_j)=4$.

On the other hand, assume that there exists a set $A'\subseteq B\cup A$ of candidates, $|A'|\le k$, such that $c$ is a unique winner of election $E'=(C-A',V)$.
Since $|A'|\le k$, there are at least $2k$ candidates from $B$ in $E'$ and so it has to be $\score{E'}(c)\ge k-1$.
However, the only way to increase $c$'s score to $k-1$ (or higher) by deleting at most $k$ candidates is to delete $k-1$ (or more) candidates from $A$.
Now, if we delete exactly $k-1$ candidates from $A$, there is some candidate $b_i\in B$ in the election, for which $\score{E'}(b_i)\ge k-1$.
Thus, $A'$ must contain exactly $k$ candidates from $A$.
Deleting these candidates increases $c$'s score to be $k$.
To ensure that the scores of the candidates in $B$ do not exceed $k$, we must ensure that $A'$ corresponds to an exact set cover of $B$ by sets from $\mathcal{S}$, so the score of each candidate in $B$ chops down to at most $k-1$.
\end{proof}

\begin{theorem} \label{th:pldcd}
	\PlControl{DC}{D} is \sharpPclass-metric-complete.
\end{theorem}

\begin{proof}
This fact follows from \sharpPclass-parsimonious-completeness of constructive version of deleting candidates in plurality election.
Similarly as in theorem~\ref{th:placd}, we define $I=(E,c,k)$ to be an instance of \PlControl{DC}{C} and we define $s_I$ to be a number of solutions to $I$.
It can be verified that the answer to \PlControl{DC}{D} for $I$ is exactly $\sum_{i=0}^k\binom{m-1}{i}-s_I$.
Thus, the problem is \sharpPclass-metric-complete, after the similar argumentation as in theorem~\ref{th:placd}.
\end{proof}

\section{Adding and Deleting Voters} \label{sec:plvot}

In control by adding voters we have a collection $V'$ containing additional voters we can use to make $c$ the unique winner.
Intuitively, in order to make $c$ the unique winner, we should add to the election voters from $V'$ that rank $c$ first, and once we have done that, we can get from $V'$ other voters, while not exceeding $k$ selections in total.
This strategy leads us to a polynomial-time algorithm which returns the number of such selections.

\begin{theorem} \label{th:plavc}
	\PlControl{AV}{C} is in \FPclass.
\end{theorem}
\begin{proof}
For each integer $i$, $0\le i\le m-1$, let $A_i$ be a multiset, such that $A_i\subseteq V'$ and it contains voters that rank $c_i$ first (we assume $c_0=c$).
We provide the algorithm which counts all the solutions, and we prove it works in polynomial time.
We use function $\id{count}(E,V',c,k,j)$ which counts how many possibilities there are, such that we can choose at most $k-j$ elements from $V'-A_0$, preserving $c$ as the unique winner of the altered election.
The pseudocode of the algorithm is presented below.

\begin{codebox}
\Procname{$\proc{Pl-AVC-Count}(E,V',c,k)$}
\li	\If $c$ is the unique winner of $E$ \label{plavc:c_checking}
\li		\Then $k_0:=0$ \label{plav:k0_init1}
\li		\Else $k_0:=\max_{c_i\in C}(\score{E}(c_i)-\score{E}(c)+1)$ \label{plavc:k0_init2}
		\End
\li	$\id{result}:=0$ \label{plavc:result_init}
\li	\For $j:=k_0$ \To $\min(|A_0|,k)$ \label{plavc:counting_loop}
\li		\Do $\id{result}:=\id{result}+\,\binom{|A_0|}{j}\cdot\id{count}(E,V',c,k,j)$ \label{plavc:counting}
		\End
\li	\Return $\id{result}$ \label{plavc:returning_result}
\end{codebox}

At the beginning, the algorithm finds the minimum number $k_0$ of elements from $A_0$ that added to $V$ will make $c$ the unique winner of election $E$.
Clearly, if $c$ already is the unique winner of $E$, then $k_0$ is 0, and otherwise $k_0$ is $\max_{c_i\in C}(\score{E}(c_i)-\score{E}(c)+1)$.
After we compute $k_0$, we loop from this value until $|A_0|$ or $k$ is reached, depending which of them is smaller, and we collect partial results.
More specifically, for each integer $j$, $k_0\le j\le\min(|A_0|,k)$, we compute the number of multisets $U$, $U\subseteq V'$, such that $U$ contains exactly $j$ voters from $A_0$, at most $k-j$ voters from $V'-A_0$, and $c$ is the unique winner of $E=(C,V\uplus U)$.
It is easy to verify that for a given $j$, there are exactly $h(j)=\binom{|A_0|}{j}\cdot\id{count}(E,V',c,k,j)$ such multisets.
Our algorithm returns $\sum_{j=k_0}^{\min(|A_0|,k)}h(j)$.
To complete the proof it suffices to show the subroutine \id{count} can be computed in polynomial time.

Let us fix $j$, $k_0\le j\le\min(|A_0|,k)$ and show how to compute $\id{count}(E,V',c,k,j)$.
Our goal is to count the number of ways in which we can add at most $k-j$ voters from $V'-A_0$ so that no candidate $c_i\in C$ has score higher than $\score{E}(c)+j-1$.
For each candidate $c_i\in C$, we can add at most
\[
	\ell_i = \min\bigl(|A_i|,j+\score{E}(c)-\score{E}(c_i)-1\bigr),
\]
voters from $A_i$; otherwise $c_i$'s score would exceed $\score{E}(c)+j-1$.

For each integer $p$, $0\le p\le k-j$, and each integer $q$, $1\le q\le m-1$, let $a_{p,q}$ be the number of multisets $U$, $U\subseteq A_1\cup A_2\cup\cdots\cup A_q$, that contain exactly $p$ voters and such that each candidate $c_1$, $c_2$,~\dots,~$c_q$ has score at most $\score{E}(c)+j-1$ in the election $E=(C,V\uplus U)$.
It can be checked that $a_{p,q}$ holds the recursion:
\[
	a_{p,q} =
	\begin{cases}
		\sum_{i=0}^{\min(\ell_q,p)}\!\binom{|A_q|}{i}a_{p-i,q-1}, & \text{if $p>0$ and $q>1$}, \\[2mm]
		1, & \text{if $p=0$ and $q>1$}, \\[1mm]
		\binom{|A_1|}{p}, & \text{if $p\le|A_1|$ and $q=1$}, \\[1mm]
		0, & \text{if $p>|A_1|$ and $q=1$}.
	\end{cases}
\]
These values can be computed in polynomial time via standard dynamic programming method.
Naturally, $\id{count}(E,V',c,k,j)=\sum_{p=0}^{k-j}a_{p,m-1}$.

As we can see, function \id{count} is polynomial-time computable, and so is the main algorithm.
\end{proof}

Using this theorem, we can easily get a result for the destructive setting.

\begin{theorem} \label{th:plavd}
	\PlControl{AV}{D} is in \FPclass.
\end{theorem}

\begin{proof}
The main observation we make, is that the solution to \PlControl{AV}{D} is equal to the number of all possible selections of at most $k$ new voters from $V'$, subtracted by the number of such selections that make $c$ the unique winner.
There are exactly $\sum_{i=0}^k\binom{|V'|}{i}$ multisets $U$ such that $U\subseteq V'$ and $|U|\le k$.
Of these, there are exactly $\proc{Pl-AVC-Count}(E,V',c,k)$ multisets of voters whose inclusion in the election ensures that $c$ is the unique winner.
Thus, there are exactly
\[
	\sum_{i=0}^k\binom{|V'|}{i}-\proc{Pl-AVC-Count}(E,V',c,k)
\]
multisets $U$ of cardinality at most $k$ containing voters from $V'$ whose inclusion in the election ensures that $c$ is not the unique winner.
Clearly, we can compute this value in polynomial time.
\end{proof}

The control by deleting voters is similar to the one where we add them, but instead of looking for additional voters, we can only remove them from the election.
It is clear that deleting voters for which $c$ is not the top-ranked candidate is a necessary action to make $c$ the winner.
We use this observation in the following theorem.

\begin{theorem} \label{th:pldvc}
	\PlControl{DV}{C} is in \FPclass.
\end{theorem}

\begin{proof}
For each integer $i$, $0\le i\le m-1$, let $A_i$ be a multiset, such that $A_i\subseteq V$ and it contains voters that rank $c_i$ first (we assume $c_0=c$).
For each integer $j$, $0\le j\le k$, we define $\id{count}(E,c,k,j)$ to be the number of multisets $U$, $U\subseteq V-A_0$ such that:
\begin{Enumerate}
	\item $|U|\le k-j$, and
	\item in election $E'=(C,V-U)$ for each candidate $c_i\in C$ it holds $\score{E'}(c_i)\le\score{E}(c)-j-1$.
\end{Enumerate}
The following algorithm returns the number of solutions to the problem.

\begin{codebox}
\Procname{$\proc{Pl-DVC-Count}(E,c,k)$}
\li	$\id{result}:=0$ \label{pldvc:result_init}
\li	\For $j:=0$ \To $\min(|A_0|,k)$ \label{pldvc:counting_loop}
\li		\Do $\id{result}:=\id{result}+\,\binom{|A_0|}{j}\cdot\id{count}(E,c,k,j)$ \label{pldvc:counting}
		\End
\li	\Return $\id{result}$ \label{pldvc:returning_result}
\end{codebox}

In each iteration of the main loop we consider deleting exactly $j$ voters from $A_0$ (there are $\binom{|A_0|}{j}$ ways to pick these $j$ voters).
Assuming we remove from $V$ exactly $j$ members of $A_0$, we must also remove some number of voters from $V-A_0$, to make sure that $c$ is the unique winner of the resulting election.
The number of ways in which this can be achieved is $\id{count}(E,c,k,j)$.
It is easy to verify that indeed our algorithm works correctly.
It remains to show how to compute $\id{count}(E,c,k,j)$.

Let us fix some value $j$, $0\le j\le\min(|A_0|,k)$.
For each $c_i\in C$, we define:
\[
    \ell_i = \max\bigl(0,j+\score{E}(c_i)-\score{E}(c)+1\bigr).
\]
Intuitively, $\ell_i$ is the minimal number of voters from $A_i$ that need to be removed from the election for $c$ to have score higher than $c_i$ (assuming $j$ voters from $A_0$ have been already removed from the election).

For each integer $p$, $0\le p\le k-j$, and each integer $q$, $1\le q\le m-1$, let $a_{p,q}$ be the number of multisets $U$, $U\subseteq A_1\cup A_2\cup\dots\cup A_q$, such that $|U|=p$ and each candidate $c_1$, $c_2$,~\dots,~$c_q$ has score at most $\score{E}(c)-j-1$ in election $E'=(C,V-U)$.
The following recursive relation holds:
\[
	a_{p,q} =
	\begin{cases}
		\sum_{i=\ell_q}^{\min(|A_q|,p)}\!\binom{|A_q|}{i}a_{p-i,q-1}, & \text{if $p\ge\ell_q$ and $q>1$}, \\[2mm]
		\binom{|A_1|}{p}, & \text{if $p\ge\ell_1$ and $q=1$}, \\[1mm]
		0, & \text{if $p<\ell_q$}.
	\end{cases}
\]
Thus, for each $p,q$ we can compute $a_{p,q}$ in polynomial time using dynamic programming techniques in polynomial time.
It is easy to see that $\id{count}(E,c,k,j)=\sum_{p=0}^{k-j}a_{p,m-1}$, which also is polynomial-time computable.
\end{proof}

As earlier, we can show that the destructive control by deleting voters can be considered as a complement to the number of possibilities in the constructive case of this problem.

\begin{theorem} \label{th:pldvd}
	\PlControl{DV}{D} is in \FPclass.
\end{theorem}

\begin{proof}
We will show it by analogy to the proof of theorem~\ref{th:plavd}.
The destructive case of deleting voters can be solved by computing the number of possible selections of at most $k$ voters, which will be removed from the election in order to make $c$ the unique winner, and by subtracting this value from the number of all possible selections of at most $k$ voters.
There are exactly $\sum_{i=0}^k\binom{n}{i}$ multisets $U$ such that $U\subseteq V$ and $|U|\le k$.
Among them there are exactly $\proc{Pl-DVC-Count}(E,c,k)$ multisets of voters whose deletion from the election ensures that $c$ is the unique winner.
Therefore, the answer to this problem is
\[
    \sum_{i=0}^k\binom{n}{i}-\proc{Pl-DVC-Count}(E,c,k),
\]
and it can obviously be produced in polynomial time.
\end{proof}

\section{Summary} \label{sec:plsum}

In this chapter we have studied counting variants of control problems under plu\-ra\-li\-ty voting system.
We showed that controlling by altering the set of candidates leads to \sharpPclass-complete problems---control by adding or deleting candidates when considering the constructive case is \sharpPclass-parsimonious-complete, while under the destructive case these types of control are \sharpPclass-metric-complete.
However, each type of control involving changes in the voters' multiset is computationally easy---we showed that these problems are in \FPclass\ and we developed algorithms that solve them in polynomial-time.
