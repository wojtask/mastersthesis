\chapter{Control Problems in Maximin Voting} \label{ch:maximin}

Maximin is the last voting system we study in this thesis.
The proofs in this chapter are motivated by the analysis of maximin in the work of Faliszewski, Hemaspaandra, and Hemaspaandra \cite{faliszewski3}.

Several proofs in this chapter use an extended notation for total orders.
Let $X=\{x_1,x_2,\dots,x_k\}$ be a set and $\mathcal{T}(X)$ be a set of all total orders over $X$.
For a fixed total order $T_X\in\mathcal{T}(X)$ such that $x_1\succ x_2\succ\cdots\succ x_k$ we write $y\succ T_X\succ z$ to denote $y\succ x_1\succ x_2\succ\cdots\succ x_k\succ z$.
Also, we define $\overleftarrow{T_X}$ to be a total order $T_X$ but in reverse, i.e., $x_k\succ x_{k-1}\succ\cdots\succ x_1$.

\section{Adding and Deleting Candidates} \label{sec:mmcan}

In the following theorem we show \sharpPclass-completeness of the counting variant of constructive adding candidates and we use the result in the next theorem concerning the destructive case.

\begin{theorem} \label{th:mmacc}
	\MmControl{AC}{C} is \sharpPclass-parsimonious-complete.
\end{theorem}

\begin{proof}
It is clear that this problem belongs to \sharpPclass.
We give a parsimonious reduction from \sharpXthreeC.
Let $(B,\mathcal{S})$, where $B=\{b_1,\dots,b_{3k}\}$ and $\mathcal{S}=\{S_1,\dots,S_r\}$, be an instance of the \sharpXthreeC\ problem.

We construct a \MmControl{AC}{C} instance as follows.
Let $E=(C,V)$ be an election, where $C=B\cup\{c\}$ is the set of registered candidates, and $V=\{v_1,\dots,v_{2r+2}\}$ is a multiset of voters.
We also have a set $A=\{a_1,\dots,a_r\}$ of spoiler (unregistered) candidates.
Each candidate $a_i$ in $A$ corresponds to a set $S_i$ in $\mathcal{S}$.
For each $S_i\in\mathcal{S}$ we fix total orders $T_{S_i}\in\mathcal{T}(S_i)$, $T_{B-S_i}\in\mathcal{T}(B-S_i)$, and $T_{A-\{a_i\}}\in\mathcal{T}(A-\{a_i\})$.
For each $S_i\in\mathcal{S}$, voter $v_i$ reports preference order $c\succ T_{B-S_i}\succ a_i\succ T_{S_i}\succ T_{A-\{a_i\}}$ and voter $v_{r+i}$ reports preference order $\overleftarrow{T_{A-\{a_i\}}}\succ a_i\succ\overleftarrow{T_{S_i}}\succ\overleftarrow{T_{B-S_i}}\succ c$.
Voter $v_{2r+1}$ reports $c\succ T_A\succ T_B$ and voter $v_{2r+2}$ reports $\overleftarrow{T_B}\succ c\succ\overleftarrow{T_A}$, where $T_A\in\mathcal{T}(A)$, and $T_B\in\mathcal{T}(B)$ are some fixed total orders.

We claim that every set $A'$, $A'\subseteq A$, such that $|A'|\le k$ and that $c$ is a unique winner of maximin election $E'=(C\cup A',V)$ corresponds one-to-one to a family $\mathcal{S'}$, $\mathcal{S'}\subseteq\mathcal{S}$, such that $|\mathcal{S'}|=k$ and $\mathcal{S'}$ is an exact set cover of $B$.

To show the claim, let us observe that for each pair of distinct elements $b_i$, $b_j\in B$, we have $\beaten{E}(b_i,b_j)=\beaten{E}(c,b_i)=\beaten{E}(b_i,c)=r+1$.
That is, the winners in election $E$ are all candidates as all of them tie.
Candidate $c$ is a winner of $E$, but we need to make him the unique winner.
Now consider a set $A'\subseteq A$, $|A'|\le k$, and an election $E'=(C\cup A',V)$.
Note that values of $\beaten{E}$ and $\beaten{E'}$ are the same for each pair of candidates in $B\cup\{c\}$.
For each pair of distinct elements $a_i$, $a_j\in A'$, we have $\beaten{E'}(c,a_i)=r+2$, $\beaten{E'}(a_i,c)=r$, and $\beaten{E'}(a_i,a_j)=r+1$.
For each $b_i\in B$ and each $a_j\in A'$ we have that
\[
	\beaten{E'}(b_i,a_j) =
	\begin{cases}
		r, & \text{if $b_i\in S_j$}, \\
		r+1, & \text{if $b_i\notin S_j$},
	\end{cases}
\]
and $\beaten{E'}(a_j,b_i)=|V|-\beaten{E'}(b_i,a_j)=2r+2-\beaten{E'}(b_i,a_j)$.
Thus, by definition of maximin, we have the following scores in $E'$:
\begin{Enumerate}
	\item $\score{E'}(c)=r+1$,
	\item for each $a_j\in A'$, $\score{E'}(a_j)=r$, and
	\item for each $b_i\in B$,
	\[
		\score{E'}(b_i) =
		\begin{cases}
			r, & \text{if there is $a_j\in A'$ such that $b_i\in S_j$}, \\
			r+1, & \text{otherwise}.
		\end{cases}
	\]
\end{Enumerate}

Since $|A'|\le k$, it is easy to see that $c$ is the unique winner of $E'$ if and only if for each $b_i\in B$, $\score{E'}(b_i)=r$, which holds if and only if family $\mathcal{S'}$ corresponding to $A'$ is an exact set cover of $B$.
\end{proof}

\begin{theorem} \label{th:mmacd}
	\MmControl{AC}{D} is \sharpPclass-metric-complete.
\end{theorem}

\begin{proof}
We proof if this theorem is analogous to the proof of Theorem~\ref{th:placd}.
The answer to this problem is $\sum_{i=0}^k\binom{|C'|}{i}$ subtracted by the answer to problem \MmControl{AC}{C} for the same instance.
Since the constructive version is \sharpPclass-parsimonious-complete, the current problem is \sharpPclass-metric-complete.
\end{proof}

Control by deleting candidates in maximin has an interesting property.
While the decision variant of this problem is polynomial-time computable, it is very likely that its counting variant is computationally hard.
We were not able to prove this fact, but we simplify \MmControl{DC}{C} by modelling it as a graph problem with a simpler structure.

First we prove that the decision variant of control by deleting candidates in maximin election is in \Pclass.

\begin{theorem} \label{th:mmdcc}
	${\it Maximin\/}\textnormal{-DC${}_\textnormal{C}$-}\textsc{Control}$ is in \Pclass.
\end{theorem}

\begin{proof}
Given an election $E=(C,V)$, a distinguished candidate $c\in C$, and a positive integer $k\le m$, we ask if there is a set $B\subseteq C$ with $|B|\le k$, such that $c$ is the unique winner of election $E'=(C-B,V)$.
Let us study how do scores of candidates change after removing candidates.
Since score of candidate $p\in C$ in maximin election is defined as $\min_{p'\in C-\{p\}}\beaten{E}(p,p')$, we cannot decrease $p$'s score.
So, there are two ways to make $c$ the unique winner:
\begin{Enumerate}
	\item by deleting such candidates $c'$ that have greater score than $c$ has, or
	\item by deleting such candidates $c''$ for which $\beaten{E}(c,c'')=\score{E}(c)$, i.e., the value $\beaten{E}(c,c'')$ is minimal for $c$.
\end{Enumerate}
Of course, when trying to increase $c$'s score by applying these rules, we may increase other candidates' scores as well, and make these candidates winners instead of $c$.
The idea is to performing these operations while preserving $c$'s score, and then to check if the number of deleted candidates does not exceed $k$.

Based on this observation we create a polynomial-time algorithm that solves problem ${\it Maximin\/}\textnormal{-DC${}_\textnormal{C}$-}\textsc{Control}$.

\begin{codebox}
\Procname{$\proc{Mm-DCC}(E,c,k)$}
\li	\For $i:=0$ \To $n$
\li		\Do Let $M_{c,i}$ be the set of candidates $c'\in C-\{c\}$, s.t.\ $\beaten{E}(c,c')=i$.
		\End
\li	\For $j:=\score{E}(c)$ \To $\max_{c'\in C-\{c\}}\beaten{E}(c,c')$ \label{mmdcc:main_loop_beg}
\li		\Do
			$B:=\bigcup_{i=0}^{j-1}M_{c,i}$
\li			\While $(C-\{c\})-B$ contains $p$, s.t. $\score{(C-B,V)}(p)\ge\score{(C-B,V)}(c)$ \label{mmdcc:greedy_beg}
\li				\Do $B:=B\cup\{p\}$
				\End \label{mmdcc:greedy_end}
\li			\If $M_{c,j}\nsubseteq B$ and $|B|\le k$
\li				\Then \Return ``yes''
				\End
		\End \label{mmdcc:main_loop_end}
\li	\Return ``no''
\end{codebox}

The first step in the algorithm is computing, for each integer $i$, $0\le i\le n$, the set $M_{c,i}=\{\,c'\in C-\{c\}:\beaten{E}(c,c')=i\,\}$.
Note that if $j\le\max_{c'\in C-\{c\}}\beaten{E}(c,c')$ and $B=\bigcup_{i=0}^{j-1}M_{c,i}$, then $c$ have score equal to $j$ in election $(C-B,V)$.

The main loop of the algorithm (lines \ref{mmdcc:main_loop_beg}--\ref{mmdcc:main_loop_end}) consists of at most $n+1$ steps.
For each integer $j$, $\score{E}(c)\le j\le\max_{c'\in C-\{c\}}\beaten{E}(c,c')$, we check if it is possible to delete at most $k$ candidates and make $c$ the unique winner having score equal to $j$ in the modified election.
To do this, we set $B$ to be $\bigcup_{i=0}^{j-1}M_{c,i}$, and then we try to add to $B$ such candidates which prevent $c$ from being the unique winner in election $(C-B,V)$.
Clearly, every candidate $p\in(C-\{c\})-B$ of score higher than or equal to $c$'s score in election $(C-B,V)$ should have been added to $B$.
We repeat this operation for a current set $B$, because the candidates just added to $B$ may have increased scores of other candidates in $(C-B,V)$.
This greedy procedure, implemented in lines \ref{mmdcc:greedy_beg}--\ref{mmdcc:greedy_end}, eventually ends.
However, now we could have $M_{c,j}\subseteq B$, which would mean that we have increased $c$'s score, but we assumed it should be constant during this iteration of the main loop.
Otherwise, if the cardinality of the set $B$ is at most $k$, the algorithm returns ``yes'', as $c$ is the unique winner of election $E'=(C-B,V)$.
After checking every value of $j$ without success the algorithm returns ``no'', because it is impossible to make $c$ the unique winner.

Clearly, our algorithm can be performed in polynomial time, thus we have that ${\it Maximin\/}\textnormal{-DC${}_\textnormal{C}$-}\textsc{Control}\in\Pclass$.
\end{proof}

Let us now move on to the counting variant.
Note that when the algorithm described in the proof above is going to return ``yes'', it has already found one possible way to delete candidates.
At this moment we may delete additional candidates from $(C-\{c\})-B$ as long as $c$ is still the unique winner with score equal to $j$ and the total number of deleted candidates does not exceed $k$.
Our goal is to count such possibilities.

When we delete a candidate from $(C-\{c\})-B$, other candidate's score may increase and become greater than or equal to $j$, in which case it should also be deleted to make sure $c$ is still a unique winner.
Sometimes, when such a situation occurs, we need to delete not only one candidate but a group of them.
More formally, candidate $p$ increases its score to be $j$ or greater after we delete each candidate $p'$ from $\bigcup_{i=0}^{j-1}M_{p,i}$.
This situation can be modeled as a directed graph $G$, in which $C-B$ is the set of vertices, and for each pair of vertices $u$ and $v$, there is an arc from $u$ to $v$ if and only if $v\in\bigcup_{i=0}^{j-1}M_{u,i}$.
Note that in order to make $v$'s score greater than or equal to $j$ (and therefore to make $v$ a new winner), we need to delete every candidate represented by vertex $u$, such that $(u,v)$ is an arc in graph $G$.

By \emph{terminal vertex} we will call a vertex $u$ such that there are no arcs leaving $u$.
Note that in our graph $G$, the only terminal vertex is $c$.
Our problem is to count such subsets of candidates from $(C-\{c\})-B$ whose cardinality does not exceed $k-|B|$ and whose deletion does not change $c$'s score (which is $j$).
For graph $G$, it means that we count sets of vertices $R$, $R\subseteq (C-\{c\})-B$, such that:
\begin{Enumerate}
	\item $|R|\le k-|B|$,
	\item $M_{c,j}\nsubseteq R$, and
	\item $c$ is the only terminal vertex of $G'$---the induced subgraph of $G$ in which the set of vertices is $(C-B)-R$.
\end{Enumerate}
The first condition above is trivial.
Without the second condition we could change the score of $c$, while our goal is to keep this score constant.
The last condition is also important---if there would be any other terminal vertex in $G'$, we should have removed it from our graph, as the candidate represented by it has score greater than or equal than $c$'s score.

Note, that we can omit the second condition when counting sets of vertices satisfying these conditions.
We create induced subgraph $G''$ of $G$ in which we removed vertices from $M_{c,j}$ and the smallest set of additional vertices which should be further removed in order to make $c$ the graph's only terminal vertex.
The set of additional vertices to remove can be computed by a greedy method from the proof of Theorem~\ref{th:mmdcc}.
The answer to our counting problem is the number of sets of vertices of $G$ satisfying the first and the third condition, decreased by the number of sets of vertices of $G''$ satisfying the first and the third condition.

The following conjectures state that control by deleting candidates in maximin (and, equivalently, the graph problem from the previous paragraphs) is computationally hard, in both constructive and destructive cases.

\begin{conjecture} \label{cj:mmdcc}
	\MmControl{DC}{C} is \sharpPclass-parsimonious-complete.
\end{conjecture}

\begin{conjecture} \label{cj:mmdcd}
	\MmControl{DC}{D} is \sharpPclass-metric-complete.
\end{conjecture}

If Conjecture~\ref{cj:mmdcc} is true, the proof of Conjecture~\ref{cj:mmdcd} will be simple adaptation of the proof of Theorem~\ref{th:pldcd}.

\section{Adding and Deleting Voters} \label{sec:mmvot}

\begin{theorem} \label{th:mmavc}
	\MmControl{AV}{C} is \sharpPclass-parsimonious-complete.
\end{theorem}

\begin{proof}
Clearly, the problem is in \sharpPclass.
We show a parsimonious reduction from \sharpXthreeC\ in order to prove it is \sharpPclass-parsimonious-hardness.

Given an instance $(B,\mathcal{S})$ of \sharpXthreeC, where $B=\{b_1,\dots,b_{3k}\}$ and $\mathcal{S}=\{S_1,\dots,S_r\}$, we construct an election $E=(C,V)$, where $C=B\cup\{c,w\}$ is the set of candidates, and $V=\{v_1,\dots,v_{4k}\}$ is a multiset of registered voters.
For a fixed total order $T_B\in\mathcal{T}(B)$ there are:
\begin{Enumerate}
	\item $2k$ voters with preference order $w\succ T_B\succ c$,
	\item $k$ voters with preference order $c\succ T_B\succ w$, and
	\item $k$ voters with preference order $c\succ w\succ T_B$.
\end{Enumerate}
We also have a multiset $V'$ of $r$ unregistered voters, where $i$th voter, $1\le i\le r$, reports preference order $T_{B-S_i}\succ c\succ T_{S_i}\succ w$, with fixed total orders $T_{S_i}\in\mathcal{T}(S_i)$ and $T_{B-S_i}\in\mathcal{T}(B-S_i)$.

We claim that every multiset $U$, $U\subseteq V'$, such that $|U|\le k$ and that $c$ is a unique winner of maximin election $E'=(C,V\uplus U)$ corresponds one-to-one to a family $\mathcal{S'}$, $\mathcal{S'}\subseteq\mathcal{S}$, such that $|\mathcal{S'}|=k$ and $\mathcal{S'}$ is an exact set cover of $B$.

It is easy to verify that for each $b_i\in B$ it holds that $\beaten{E}(c,b_i)=\beaten{E}(c,w)=2k$.
Similarly, for each $b_i\in B$ it holds that $\beaten{E}(w,b_i)=3k$ and that $\beaten{E}(w,c)=2k$.
Thus, $\score{E}(c)=\score{E}(w)=2k$.
It is also easy to verify that for each $b_i\in B$ it holds that $\score{E}(b_i)\le k$.

Let $U$, $U\subseteq V'$, be a multiset containing at most $k$ voters, and let $E'=(C,V\uplus U)$.
For each $b_i\in B$ it holds that $\score{E'}(b_i)\le2k$.
Since each voter in $V'$ ranks $w$ as the least desirable candidate, $\score{E'}(w)=2k$.
Let us now find a score of $c$.
If there exists a candidate $b_i\in B$ such that there is no voter in $V'$ that prefers $c$ to $b_i$, then $\beaten{E'}(c,b_i)=2k$ and thus $\score{E'}(c)=2k$.
Otherwise, $\score{E'}(c)\ge2k+1$.
Thus, $c$ is a unique winner of $E'$ if and only if $U$ corresponds to an exact set cover of $B$.
\end{proof}

\begin{theorem} \label{th:mmdvc}
	\MmControl{DV}{C} is \sharpPclass-parsimonious-complete.
\end{theorem}

\begin{proof}
It is clear this problem is in \sharpPclass.
We show that it is \sharpPclass-parsimonious-hard by parsimonious transformation from \sharpXthreeC.

Let $(B,\mathcal{S})$ be an instance of the \sharpXthreeC\ problem, where $B=\{b_1,\dots,b_{3k}\}$ with $k\ge3$, and $\mathcal{S}=\{S_1,\dots,S_r\}$.
We form an election $E=(C,V)$ where $C=B\cup\{c,w\}$ and $V=V'\cup V''$ with $V'=\{v_1',\dots,v_{2r}'\}$, $V''=\{v_1'',\dots,v_{2r-k+2}''\}$.
For each $i$, $1\le i\le r$, and for fixed total orders $T_{S_i}\in\mathcal{T}(S_i)$, $T_{B-S_i}\in\mathcal{T}(B-S_i)$, voter $v_i'$ reports preference order $w\succ T_{B-S_i}\succ c\succ T_{S_i}$ and voter $v_{r+i}'$ reports preference order $w\succ\overleftarrow{T_{S_i}}\succ c\succ\overleftarrow{T_{B-S_i}}$.
Let $T_B\in\mathcal{T}(B)$ be a fixed total order.
Among the voters in $V''$ there are:
\begin{Enumerate}
	\item 2 voters with preference order $c\succ w\succ T_B$,
	\item $r-k$ voters with preference order $c\succ T_B\succ w$, and
	\item $r$ voters with preference order $T_B\succ c\succ w$.
\end{Enumerate}

We claim that every multiset $U$, $U\subseteq V$, such that $|U|\le k$ and that $c$ is a unique winner of maximin election $E'=(C,V-U)$ corresponds one-to-one to a family $\mathcal{S'}$, $\mathcal{S'}\subseteq\mathcal{S}$, such that $|\mathcal{S'}|=k$ and $\mathcal{S'}$ is an exact set cover of $B$.

It is easy to see that candidates in election $E$ have the following scores:
\begin{Enumerate}
	\item $\score{E}(w)=2r$ (because $\beaten{E}(w,c)=2r$ and for each $b_i\in B$, $\beaten{E}(w,b_i)=2r+2$),
	\item $\score{E}(c)=2r-k+2$ (because for each $b_i\in B$, $\beaten{E}(c,w)=\beaten{E}(c,b_i)=2r-k+2$), and
	\item for each $b_i\in B$, $\score{E}(b_i)\le2r-k$ (because $\beaten{E}(b_i,w)=2r-k$).
\end{Enumerate}
Before we delete any voter, $w$ is the unique winner with $k-2$ more points than $c$.
We can decrease $w$'s score by at most $k$ points via deleting at most $k$ voters.

Let $U$, $U\subseteq V$, be a multiset of voters such that $c$ is the unique winner of $E'=(C,V-U)$.
We partition $U$ into $U'\cup U''$, where $U'=U\cap V'$ and $U''=U\cap V''$.
We claim that $U''$ is empty.
Let us assume that $U''\ne\emptyset$ and let $E''=(C,V-U'')$.
Since every voter in $V''$ prefers $c$ to $w$, we have that $\beaten{E''}(c,w)=\beaten{E}(c,w)-|U''|$ and thus $\score{E''}(c)\le\score{E}(c)-|U''|$.
Similarly, assuming $U''\ne\emptyset$, it is easy to see that $\score{E''}(w)\ge\score{E}(w)-|U''|+1$.
It holds after the observation that deleting any single member of $U''$ cannot decrease $w$'s score.
That is, we have that $\score{E''}(c)\le2r-k+2-|U''|$ and $\score{E''}(w)\ge2r+1-|U''|$.
So in $E''$, $w$ has at least $k-1$ more points than $c$.
Since $|U''|\ge1$, we can delete at most $k-1$ voters $U'$ from election $E''$.
But then $c$ will not be a unique winner of $E'$, which is a contradiction.

Thus, $U$ contains members of $V'$ only.
Since $w$ is ranked first in every vote in $V'$, deleting voters from $U$ decreases $w$'s score by exactly $|U|$.
Further, deleting voters $U$ certainly decreases $c$'s score by at least one point.
Thus, after deleting voters $U$ we have:
\begin{Enumerate}
	\item $\score{E'}(w)=2r-|U|$, and
	\item $\score{E'}(c)\le2r-k+2-1=2r-k+1$.
\end{Enumerate}
In consequence, the only possibility that $c$ is a unique winner after deleting voters $U$ is that $|U|=k$ and we have equality in item~2 above.
It is easy to verify that this equality holds if and only if $U$ contains $k$ voters among $v_1',\dots,v_r'$ that correspond to an exact set cover of $B$ via sets from $\mathcal{S}$ (under the assumption that $k\ge3$).
\end{proof}

\begin{theorem} \label{th:mmadvd}
	\MmControl{AV}{D} and \MmControl{DV}{D} are \sharpPclass-me\-tric-complete.
\end{theorem}

\begin{proof}
We get the result by applying the ideas from theorems~\ref{th:acavd} and~\ref{th:acdvd} to maximin system.
\end{proof}

\section{Summary} \label{sec:mmsum}

As we could see in this chapter, control in maximin system is computationally hard in every considered case (in case of deleting candidates we only conjecture that this result holds).
Thus, each studied variant of the winner prediction problem is computationally hard in maximin.
This fact is interesting also because of the fact that we proved there exists a polynomial-time algorithm that solve the decision variant of control by deleting candidates.
However, it is very likely there are no polynomial-time algorithm that solves the counting variant of such control type.
We leave the proof of this fact for future research.
